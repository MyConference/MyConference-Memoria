\chapter{Tecnolog�as}

\section{Heroku}
\label{tech:heroku}

\begin{center}
\imagen{Bitmap/Tecnologias/heroku}{width=0.5\textwidth}
\end{center}

\textit{Heroku} es una plataforma como servicio que soporta distintos lenguajes de programaci�n. \url{https://www.heroku.com/}

\textit{Plataforma como servicio} (PaaS - Platform as a Service) permite al desarrollador desplegar aplicaciones en la nube (Cloud Computing) de manera sencilla y r�pida sin tener que administrar el hardware, es decir, sin tener que instalar software adicional en la m�quina. Suele identificarse como una evoluci�n del Software como servicio (SaaS - Software as a Service), aunque la diferencia est� en que SaaS es un modelo que permite la distribuci�n del software mientras que PaaS distribuye el software y todo lo necesario para su ejecuci�n. Una de las ventajas del PaaS es el mejor aprovechamiento de las m�quinas al estar centradas en un servicio en concreto. La desventaja est� en que los servicios que proporcionan pueden quedarse escasos en algunas ocasiones. Otro ejemplo es Google App Engine.

\textit{Heroku} inicialmente solo soportaba la ejecuci�n de aplicaciones web desarrolladas en Ruby, pero posteriormente se extendi� el soporte a Java, Node.js, Scala, Clojure, Python y PHP.

Una de las grandes ventajas de \textit{Heroku} es poder obtener un Dyno (unidad de computaci�n que ejecuta una aplicaci�n) de manera gratuita. De esta manera se puede correr una aplicaci�n web que consuma pocos recursos sin ning�n tipo de gasto. Este sistema de Dynos es totalmente escalable, de tal manera que si la aplicaci�n crece y necesita m�s recursos es tan sencillo como ampliar la capacidad de los Dynos que necesita la aplicaci�n.

\textit{Heroku} dispone de una gran cantidad de extensiones (add-ons) que a�aden funcionalidad extra a las aplicaciones. Por ejemplo, la extensi�n MongoLab permite el uso de bases de datos hechas con MongoDB.

Para el desarrollo de MyConference hemos usado dos Dynos gratuitos de Heroku, uno para la ejecuci�n de la API y otro para la Web. De esta manera hemos obtenido de manera gratuita un servicio en el que poder desarrollar la aplicaci�n y poder testearla de manera sencilla y r�pida.

\section{Node.js}
\label{tech:nodejs}

\begin{center}
\imagen{Bitmap/Tecnologias/nodejs}{width=0.4\textwidth}
\end{center}

\textit{Node.js} es un entorno de programaci�n basado en JavaScript. Es la tecnolog�a usada para el desarrollo tanto de la API como de la Web. \url{http://nodejs.org/}

Dispone de una gran cantidad de m�dulos que a�aden funcionalidad al entorno. Algunos ejemplos de estos m�dulos son: 
\begin{itemize}
  \item Express: framework para el desarrollo de aplicaciones web.
  \item Winston: permite el guardado de todos los logs de una aplicaci�n.
  \item Path: para el manejo de rutas dentro de los directorios de la aplicaci�n.
  \item Mongoose: permite el manejo de manera sencilla de bases de datos hechas con MongoDB.
\end{itemize}

Estos m�dulos se pueden instalar de manera r�pida con npm, gestor oficial de paquetes de Node.js. A partir de la versi�n 0.6.3 de \textit{Node.js}, npm se instala autom�ticamente junto con el entorno. Su uso es muy simple, tan solo hay que usar el siguiente comando: 

\begin{lstlisting}[language=bash]
  $ npm install package-name
\end{lstlisting}

\section{Jade}
\label{tech:jade}

\begin{center}
\imagen{Bitmap/Tecnologias/jade}{width=0.3\textwidth}
\end{center}

\textit{Jade} es un motor de plantillas de alto rendimiento implementado en JavaScript para Node.js. Tiene una sintaxis limpia, sensible a los espacios en blanco para escribir c�digo HTML. \url{http://jade-lang.com/}

La instalaci�n de \textit{Jade} se hace mediante npm, el gestor de paquetes de Node.js: 

\begin{lstlisting}[language=bash]
  $ npm install jade
\end{lstlisting}

\begin{center}
\begin{tabular}{| l |}
\hline
HTML \\ \hline
\begin{lstlisting}
<!DOCTYPE html>
<html lang="en">
  <head>
    <title>Jade</title>
    <script type="text/javascript">
      if (foo) {
        bar(1 + 5)
      }
    </script>
  </head>
  <body>
    <h1>Jade - node template engine</h1>
    <div id="container" class="col">
      <p>You are amazing</p>
      <p>Jade is a terse and simple
         templating language with a
         strong focus on performance
         and powerful features.</p>
    </div>
  </body>
</html>
\end{lstlisting} \\ \hline
\hline
Jade \\ \hline
\begin{lstlisting}
doctype html
html(lang="en")
  head
    title= pageTitle
    script(type='text/javascript').
      if (foo) {
         bar(1 + 5)
      }
  body
    h1 Jade - node template engine
    #container.col
      if youAreUsingJade
        p You are amazing
      else
        p Get on it!
      p.
        Jade is a terse and simple
        templating language with a
        strong focus on performance
        and powerful features.
\end{lstlisting} \\ \hline 
\end{tabular}
\end{center}

\section{SASS/SCSS}
\label{tech:sass}

\begin{center}
\imagen{Bitmap/Tecnologias/sass}{width=0.3\textwidth}
\end{center}

\textit{Sass} (Syntactically Awesome Stylesheets) es un lenguaje de hojas de estilos, una extensi�n de CSS al que a�ade mucha funcionalidad, como por ejemplo el uso de variables, reglas anidadas, mixins e imports inline. \url{http://sass-lang.com/}

Los archivos \textit{Sass} suelen llevar la extensi�n .scss, pero tambi�n se puede utilizar la extensi�n .sass, aunque es menos com�n.

Para la utilizaci�n de estos ficheros, primero hay que compilarlos a archivos de CSS, no se pueden usar directamente los archivos .scss.

\begin{figure}[h]
\begin{center}
\includegraphics[width=0.5\textwidth]%
                {Imagenes/Bitmap/Tecnologias/sass-compile}
\caption{Compilaci�n de \textbf{Sass}.}
\end{center}
\end{figure}

\begin{figure}[h]
\begin{center}
\includegraphics[width=0.5\textwidth]%
                {Imagenes/Bitmap/Tecnologias/sass-to-css}
\caption{Ejemplo de c�digo escrito en \textbf{Sass} traducido a \textbf{Css}.}
\end{center}
\end{figure}

Una alternativa conocida a \textit{Sass} es Less.

\section{Bootstrap}
\label{tech:bootstrap}

\begin{center}
\imagen{Bitmap/Tecnologias/bootstrap}{width=0.5\textwidth}
\end{center}

\textit{Bootstrap} es uno de los frameworks front-end m�s populares para el desarrollo de aplicaciones web adaptables. \url{http://getbootstrap.com/}

El dise�o web adaptable (Responsive Web Design) consigue adaptar cualquier aplicaci�n web a cualquier tipo de pantalla.

\section{MongoDB}
\label{tech:mongodb}
