\chapter{Tecnolog�as}

\section{Heroku}


https://www.heroku.com/

Heroku es una plataforma como servicio que soporta distintos lenguajes de programaci�n.

Plataforma como servicio (PaaS - Platform as a Service) permite al desarrollador desplegar aplicaciones en la nube (Cloud Computing) de manera sencilla y r�pida sin tener que administrar el hardware, es decir, sin tener que instalar software adicional en la m�quina. Suele identificarse como una evoluci�n del Software como servicio (SaaS - Software as a Service), aunque la diferencia est� en que SaaS es un modelo que permite la distribuci�n del software mientras que PaaS distribuye el software y todo lo necesario para su ejecuci�n. Una de las ventajas del PaaS es el mejor aprovechamiento de las m�quinas al estar centradas en un servicio en concreto. La desventaja est� en que los servicios que proporcionan pueden quedarse escasos en algunas ocasiones. Otro ejemplo es Google App Engine.

Heroku inicialmente solo soportaba la ejecuci�n de aplicaciones web desarrolladas en Ruby, pero posteriormente se extendi� el soporte a Java, Node.js, Scala, Clojure, Python y PHP.

Una de las grandes ventajas de Heroku es poder obtener un Dyno (unidad de computaci�n que ejecuta una aplicaci�n) de manera gratuita. De esta manera se puede correr una aplicaci�n web que consuma pocos recursos sin ning�n tipo de gasto. Este sistema de Dynos es totalmente escalable, de tal manera que si la aplicaci�n crece y necesita m�s recursos es tan sencillo como ampliar la capacidad de los Dynos que necesita la aplicaci�n.

Heroku dispone de una gran cantidad de extensiones (add-ons) que a�aden funcionalidad extra a las aplicaciones. Por ejemplo, la extensi�n MongoLab permite el uso de bases de datos hechas con MongoDB.

Para el desarrollo de MyConference hemos usado dos Dynos gratuitos de Heroku, uno para la ejecuci�n de la API y otro para la Web. De esta manera hemos obtenido de manera gratuita un servicio en el que poder desarrollar la aplicaci�n y poder testearla de manera sencilla y r�pida.


\section{Node.js}

http://nodejs.org/

Node.js es un entorno de programaci�n basado en JavaScript. Es la tecnolog�a que hemos usado para el desarrollo tanto de la API como de la Web.

Dispone de una gran cantidad de m�dulos que a�aden funcionalidad al entorno. Algunos ejemplos de estos m�dulos son: 
Express: framework para el desarrollo de aplicaciones web.
Winston: permite el guardado de todos los logs de una aplicaci�n.
Path: para el manejo de rutas dentro de los directorios de la aplicaci�n.
Mongoose: permite el manejo de manera sencilla de bases de datos hechas con MongoDB.

Estos m�dulos se pueden instalar de manera r�pida con npm, gestor oficial de paquetes de Node.js. A partir de la versi�n 0.6.3 de Node.js, npm se instala autom�ticamente junto con el entorno. Su uso es muy simple, tan solo hay que usar el siguiente comando:
npm install package-name

\section{Jade}


http://jade-lang.com/

Jade es un motor de plantillas de alto rendimiento implementado en JavaScript para Node.js. Tiene una sintaxis limpia, sensible a los espacios en blanco para escribir c�digo HTML.

La instalaci�n de Jade se hace mediante npm, el gestor de paquetes de Node.js:
npm install jade


HTML
Jade

\begin{lstlisting}
<!DOCTYPE html>
<html lang="en">
  <head>
    <title>Jade</title>
    <script type="text/javascript">
      if (foo) {
        bar(1 + 5)
      }
    </script>
  </head>
  <body>
    <h1>Jade - node template engine</h1>
    <div id="container" class="col">
      <p>You are amazing</p>
      <p>Jade is a terse and simple
         templating language with a
         strong focus on performance
         and powerful features.</p>
    </div>
  </body>
</html>
\end{lstlisting}

\begin{lstlisting}
doctype html
html(lang="en")
  head
    title= pageTitle
    script(type='text/javascript').
      if (foo) {
         bar(1 + 5)
      }
  body
    h1 Jade - node template engine
    #container.col
      if youAreUsingJade
        p You are amazing
      else
        p Get on it!
      p.
        Jade is a terse and simple
        templating language with a
        strong focus on performance
        and powerful features.
\end{lstlisting}


\section{SASS/SCSS}


http://sass-lang.com/

Sass (Syntactically Awesome Stylesheets) es un lenguaje de hojas de estilos, una extensi�n de CSS al que a�ade mucha funcionalidad, como por ejemplo el uso de variables, reglas anidadas, mixins e imports inline.

Los archivos Sass suelen llevar la extensi�n .scss, pero tambi�n se puede utilizar la extensi�n .sass, aunque es menos com�n.

Para la utilizaci�n de estos ficheros, primero hay que compilarlos a archivos de CSS, no se pueden usar directamente los archivos .scss.




Algunas alternativas conocidas a Sass son less y scss.

\section{Bootstrap}


http://getbootstrap.com/

Bootstrap es uno de los front-end framework m�s populares para el desarrollo de aplicaciones web adaptables.

El dise�o web adaptable (Responsive Web Design) consigue adaptar cualquier aplicaci�n web a cualquier tipo de pantalla.


