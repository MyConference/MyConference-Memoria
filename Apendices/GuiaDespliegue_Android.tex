\section{App para Android}

La app de MyConference, descrita en el cap�tulo \ref{chap:android}, es la encargada de mostrar todos los datos introducidos a partir de la web.

Para su desarrollo solo hace falta descargarse el proyecto del repositorio alojado en Git \cite{AndroidGit}.

El IDE utilizado para este proyecto ha sido Eclipse. Basta con importar el proyecto al \textit{workspace} y autom�ticamente lo compilar�. Dentro del repositorio ya se incluyen las librer�as necesarias para su ejecuci�n.

Para testear la app es necesario un tel�fono Android con versi�n 2.3 o superior, ya que el emulador disponible con el SDK no puede ejecutar nada relacionado con Google Maps.

La aplicaci�n hace uso de algunos valores que pueden cambiar dependiendo de c�mo se haya desplegado la API en la secci�n \ref{sec:deploy-api}.
Por ello debemos modificar apropiadamente los valores de los siguientes campos en la clase \linebreak \texttt{es.ucm.myconference.util.Constants}:

\begin{description}
  \item[\texttt{APP\_ID}]: Identificador de la aplicaci�n android obtenido en el despliegue de la API.
  \item[\texttt{CONFS\_URL}]: URL base para el acceso a la API (por ejemplo,\linebreak \texttt{http://api.myconference.com}).
\end{description}

Una vez realizados estos cambios, la aplicaci�n debe compilarse en modo producci�n. Para ello, hacemos click derecho en el proyecto y, dentro de la opci�n de men� espec�fica para Android, elegimos \textit{Exportar APK firmado}.
Se pedir� un certificado para firmar la aplicaci�n. Este certificado es fundamental y debe ser siempre el mismo para que Google Play nos permita actualizar la aplicaci�n. 