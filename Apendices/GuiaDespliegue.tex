\chapter{Gu�a de Despliegue}

Para poder continuar con el desarrollo o desplegar el proyecto en su forma actual, hace falta realizar una serie de pasos en cada una de las piezas que forman MyConference.

Tanto para la API como para la Web es necesario disponer de un servidor web capaz de ejecutar aplicaciones escritas para Node.js (V�ase \ref{tech:nodejs}), e idealmente ambas usar�n diferentes servidores.
Una de las opciones m�s sencillas es utilizar alguna \textit{Plataforma como Servicio}, como por ejemplo Heroku (V�ase \ref{tech:heroku}), pero tambi�n es posible configurarlo para funcionar en un servidor privado.
Para ambas aplicaciones es necesario disponer de una base de datos MongoDB (V�ase \ref{tech:mongodb}).
Aunque las colecciones usadas son disjutas y por tanto t�cnicamente es posible usar una �nica base de datos, es conveniente por motivos de seguridas utilizar dos bases de datos independientes, una para la Web y otra para la API.

Para la app de Android solo es necesario disponer de un IDE compatible con Android, como por ejemplo Eclipse (V�ase \ref{tech:eclipse}).