\chapter{Gu�a de Despliegue}

Para poder continuar con el desarrllo o desplegar el proyecto en su forma actual, hace falta realizar una serie de pasos en cada una de las piezas que forman MyConference.

Tanto para la API como para la Web es necesario disponer de un servidor web capaz de ejecutar aplicaciones escritas para \nameref{tech:nodejs} (V�ase \ref{tech:nodejs}), e idealmente ambas usar�n diferentes servidores. Una de las opciones m�s sencillas es utilizar alguna \textit{Plataforma como Servicio}, como por ejemplo \nameref{tech:heroku} (V�ase \ref{tech:heroku}), pero tabi�n es posible configurarlo para funcionar en un servidor privado. Para ambas aplicaciones es necesario disponer de una base de datos \nameref{tech:mongodb} (V�ase \ref{tech:mongodb}). Aunque las colecciones usadas son disjutas y por tanto t�cnicamente es posible usar una �nica base de datos, es conveniente por motivos de seguridas utilizar dos bases de datos independientes, una para la Web y otra para la API.

% TODO: Sam, consideraciones generales para montar la app android (hacefalta X,hace falta Y, blah, sin detalles)