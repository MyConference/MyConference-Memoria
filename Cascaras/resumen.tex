%---------------------------------------------------------------------
%
%                      resumen.tex
%
%---------------------------------------------------------------------
%
% Contiene el cap�tulo del resumen.
%
% Se crea como un cap�tulo sin numeraci�n.
%
%---------------------------------------------------------------------

\chapter{Resumen}
\cabeceraEspecial{Resumen}

\textit{MyConference} es un conjunto de aplicaciones orientadas a la gesti�n de congresos. Su objetivo es servir como punto com�n de gesti�n y visualizaci�n de un congreso tanto a los organizadores como a los asistentes. A los organizadores les permite disponer de una plataforma en la que introducir toda la informaci�n necesaria para compartirla con los asistentes, ya sean ponentes de las conferencias o el p�blico que asiste al evento. Y a los asistentes les permite tener en su bolsillo una herramienta que pueden consultar en cualquier momento para saber d�nde ir, cual es la siguiente conferencia o de qu� temas van a tratar las ponencias para decidir a cu�les ir.

Para conseguir todo esto hemos creado una aplicaci�n para cada tipo de usuario de un congreso: un formulario web para los organizadores y una aplicaci�n m�vil para los asistentes. Ambas con los mismos principios de sencillez y claridad en su presentaci�n y uso.

\vskip 10cm

Palabras clave: congresos, gesti�n, aplicaci�n, servicio web, p�gina web, servidor, interfaz, programaci�n, aplicaciones, ipa, m�vil, android, mongo, node, jade, java.

\chapter{Abstract}
\cabeceraEspecial{Abstract}

Something about conferences

\endinput
